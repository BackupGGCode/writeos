\chapter{写在前面的话} \label{fore}


\begin{kaitext}

本书起源于中国电子工业出版社出版的一本书:《自己动手写操作系统》(于渊著)。我对《自己动手写操作系统》这本书中使用商业软件做为演示平台比较惊讶,因为不是每个人都买得起正版软件的,尤其是穷学生。我想《自》所面向的主要受众也应该是学生,那么一本介绍只有商业软件才能实现的编程技巧的书将会逼着穷学生去使用盗版,这是非常罪恶的行为~\frownie。

由于本人是一个~Linux~用户,一个开源软件的拥护者,所以就试着使用开源软件实现这本书中的所有~demo~,并在\href{http://blog.solrex.cn}{自己的博客}上进行推广。后来我觉得,为什么我不能自己写本书呢?这样我就能插入漂亮的插图,写更详尽的介绍而不怕篇幅过长,更容易让读者接受也更容易传播,所以我就开始写这本《~\BookName~》。

定下写一本书的目标毕竟不像写一篇博客,我将尽量详尽的介绍我使用的方法和过程,以图能让不同技术背景的读者都能通畅地完成阅读。但是自己写并且排版一本书不是很轻松的事情,需要耗费大量时间,所以我只能抽空一点一点的将这本书堆砌起来,这也是您之所以在本书封面看到本书版本号的原因~\smiley。

本书的最终目标是成为一本大学“计算机操作系统”课程的参考工具书,为学生提供一个~step by step~的引导去实现一个操作系统。这不是一个容易实现的目标,因为我本人现在并不自信有那个实力了解操作系统的方方面面。但是我想,立志百里行九十总好过于踯躅不前。

《自己动手写操作系统》一书开了个好头,所以在前面部分,我将主要讨论使用开源软件实现《自》的~demo~。如果您有《自》这本书,参考阅读效果会更好,不过我将尽我所能在本书中给出清楚的讲解,尽量使您免于去参考《自》一书。

出于开放性和易编辑性考虑,本书采用~\LaTeX~排版,在成书前期由于专注于版面,代码比较杂乱,可读性不强,暂不开放本书~\TeX~源代码下载。但您可以通过~SVN check out~所有本书相关的源代码和图片,具体方法请参见电子书主页。

如果您在阅读过程中有什么问题,发现书中的错误,或者好的建议,欢迎您使用我留下的联系方式与我联系,本人将非常感谢。
\vskip 1cm
\noindent
\makebox[\textwidth][r]{杨文博}
\makebox[\textwidth][r]{个人主页:\url{http://solrex.cn}}
\makebox[\textwidth][r]{个人博客:\url{http://blog.solrex.cn}}
\makebox[\textwidth][r]{2008~年~1~月~9~日}
\end{kaitext}

\begin{lined}{\textwidth}
\textbf{更新历史}
\small
\begin{description}
    \item[Rev. 1]~确定书本排版样式,添加第一章,第二章。
    \item[Rev. 2]~添加第三章保护模式。
\end{description}
\vspace{2ex}
\end{lined}

